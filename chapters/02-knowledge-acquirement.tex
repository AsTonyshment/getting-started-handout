\documentclass[../main.tex]{subfiles}

\begin{document}

\chapter{搜索和信息获取} % Chapter Over

在大学,上课和课本固然是一种重要的信息获取方式。但是课程和课本本身由于是静态的,往往无法及时更新最新的信息;然而对计算机科学等发展迅速、信息爆炸、对技术要求较高的科目,仅凭借课本等静态资源显然是远远不够的。因此,我们需要借助其他的方式来获取信息。

\section{搜索}

\subsection{搜索引擎的选择}

国内最常见的搜索引擎是百度。但是当我们在百度搜索相关内容时,第一页往往会被大量的广告占据。这显然并不是我们想要的结果。其他的国内搜索引擎都或多或少有相关的问题,因此并不好用。

由于现在大多数新购整机都预装了Windows正版系统,因此基本上都自带一个内置浏览器 Microsoft Edge。Edge 的默认搜索引擎是必应(Bing),在搜索的时候我们可以在页面顶端发现“国内版”和“国际版”的选项。在使用国内版搜索时,仍然会出现少量的广告和不相关的内容,但是相对百度而言,必应的搜索结果要好得多。在使用国际版搜索的时候,必应的搜索结果会更好,但是由于网络问题,可能会出现无法访问的情况。

细心的同学可能会发现,我们从国内网络访问Bing,无论是国内版还是国际版,网址都是cn.bing.com。而真正的Bing的网址是www.bing.com。有条件能够访问这一网址的同学可以使用这个Bing。而Google作为全球最大的国际搜索引擎,搜索结果通常比必应还要准确、直接且全面。

因此,不使用特殊方式上网的情况下,如果我们要搜索的信息\textbf{非中文社区独有},我们更推荐使用必应的国际版搜索引擎。在该课程中不将涉及任何特殊上网方式的教学。

\subsection{搜索技巧}

有时候我们搜索的时候无法搜索到想要的信息。这时候我们需要使用一些技巧。

\textbf{关键词搜索}是最常见的搜索技巧之一。我们使用完整句子进行搜索的时候,搜索引擎会利用语言模型将其拆分成多个关键词进行搜索,而语言模型总会导致一定的偏差。因此,我们可以一步到位,使用关键词进行搜索。例如,我们如果想要搜索“我怎样改善睡眠质量”,可以把它拆分成“改善睡眠 方法”关键词进行搜索;如果需要进一步约束(例如我希望方法快速起效),可以搜索“改善睡眠 \ 方法 \ 快速”。

\textbf{使用英文}是另一个常见的搜索技巧。中文互联网的一大特点是信息向应用内部收缩,形成无法被搜索引擎检索到的“深网”,导致中文开放互联网的信息量小于英文开放互联网的信息量。使用英文搜索的另一个原因是英语依然是世界上最通用的语言,尤其在技术、科学等领域,大部分的文献、资料、教程、说明等都是用英文写的;相关领域的研究材料往往也先以英文发表。因此我们在搜索的时候,使用英文搜索往往能够得到更好的结果。

即便英文水平一般的同学也不必担心。我们可以使用翻译软件(例如微软翻译、有道翻译等)将中文翻译成英文,然后再进行搜索。

\textbf{使用高级搜索选项}也是一种搜索技巧,最常见的高级搜索选项有:
\begin{itemize}
    \item 使用引号将关键词括起来,这样搜索引擎就会强制将其视为一个整体进行搜索,而不是将其拆分成多个关键词。依然以改善睡眠为例,可以使用“如何改善睡眠质量”进行搜索;
    \item 使用减号将不需要的关键词排除在外。例如,我们想要改善睡眠,但是不想看到关于药物的信息,可以使用“改善睡眠 \ 方法 \ 快速 \ -药物”进行搜索,这样搜索引擎就会把含有药物的信息排除在外;
    \item 使用“site:”限制搜索范围。例如,我们如果想要搜索“如何改善睡眠质量”,但是只想看到来自知乎的信息,可以使用“改善睡眠 \ 方法 \ 快速 \ site:zhihu.com”进行搜索。
\end{itemize}

\textbf{判断信息的可靠性}虽然不属于搜索技巧,但是却是一个非常重要的技能。我们在搜索到信息的时候,往往需要判断其可靠性。我们可以从以下几个方面来判断信息的可靠性:
\begin{itemize}
    \item 来源:信息的来源是否可靠?是否来自权威机构、专家或者知名网站?
    \item 时间:信息是否及时?是否过时?
    \item 评价:其他人对该信息的评价如何?是否有很多人认可?
    \item 完整性:信息是否完整?是否有遗漏?
    \item 可验证性:信息是否可以被验证?是否有相关的证据?
\end{itemize}

\section{信息平台}

除了使用搜索引擎在信息平台上搜索以外,我们还可以直接在著名的信息平台上面寻找相关信息。

\subsection{官方文档、官方Wiki、官方论坛}

如果我们希望获取某软件等的信息,最好的地方往往是其官方文档;对于类似于Arch Linux这种纯由社区维护的项目,其官方Wiki与论坛也是获取信息的最佳选择之一。官方文档虽然可能存在晦涩、难懂、省略等问题,但是往往依然是最权威、最全面的文档,这将会是你学习一门新技术的最佳选择。

如果你在请求问题的时候,遇到了诸如“RTFM”(Read The F**king Manual)的回应,这说明回答者认为你需要搜索官方文档和使用手册。当然在这种情况下,\textbf{他大概率是对的,你应该去读一读。}同样道理的还有STFW(Search The F**king Web)和RTFSC(Read The F**king Source Code)。而往往通过这种方式搜索信息,你能够学到的内容比直接告诉你答案要多得多。

\subsection{Stack Overflow}

堆栈溢出(Stack Overflow)是一个程序员问答网站,专门用于解决编程和技术问题。它是一个社区驱动的网站,用户可以在上面提问和回答问题。堆栈溢出有一个强大的搜索功能,可以帮助用户快速找到相关的问题和答案。从我的个人使用体验而言,这东西有点像百度贴吧和知乎的结合体,且专业性比两者都要强得多。

\subsection{GitHub}

GitHub是一个代码托管平台,用户可以在上面存储和分享代码。GitHub上有很多开源项目,用户可以在上面找到相关的代码和文档。GitHub还提供了一个强大的搜索功能,可以帮助用户快速找到相关的项目和代码。同时,GitHub也是一个非常重要的开源社区,当你对某个项目有疑问或者发现Bug的时候,你可以对该项目提出Issue,只要项目没“死”,总会有人告诉你答案;当你想要为某一项目做出贡献的时候,你可以Fork该项目,然后提交Pull Request。

\subsection{Wikipedia}

维基百科是一个自由的百科全书,用户可以在上面找到各种各样的信息。维基百科是一个社区驱动的网站,用户可以在上面编辑和修改条目。维基百科的内容是由志愿者编写和维护的,因此它的准确性和可靠性可能较低,不过它仍然是一个非常有用的信息来源。维基百科的搜索功能也很强大,可以帮助用户快速找到相关的条目。

\subsection{其他著名博客和教程}

\textbf{W3Schools}提供了许多关于开发的教程和示例,适合初学者使用。它的内容覆盖了HTML、CSS、JavaScript、SQL等多个领域。国内也有类似的网站,例如菜鸟教程、W3School等,只是内容丰富程度上较为逊色。

\textbf{OI Wiki、CTF Wiki、HPC Wiki}是一些关于算法、数据结构、编程竞赛等方面的Wiki,适合对这些领域感兴趣的同学使用。它们的内容覆盖了算法、数据结构、编程竞赛等多个领域。这些Wiki则是由在相关领域耕耘多年的选手前辈们维护的,内容质量较高。

\textbf{CS自学指南}是由信科的一位学长发起、旨在帮助计算机专业的同学自学计算机科学的一个项目。它的内容覆盖了计算机科学的各个领域,包括计算机网络、操作系统、编译原理等。它的内容质量较高,适合对计算机科学感兴趣且希望自学的同学使用。

\subsection{国内优质平台}

我们一般认为国内能够算上优质平台的有:博客园、哔哩哔哩、知乎、简书。这些平台普遍是免费的,你可以找到许多关于技术、编程、科学等方面的文章和视频。它们的内容质量参差不齐,也不乏卖课的(例如我曾经在B站看到过“预测2025年将会淘汰的编程语言:C/C++、Java、C\#、Golang、Python”等视频,当然这显然是胡扯),但是它们仍然是一个非常有用的信息来源。我们在接受信息的时候,仍然需要判断其可靠性。

特别说明:CSDN上虽然也有不少信息,但是该平台质量较低,商业化程度较高。这导致在该平台寻找信息的时候,我们必须在海量的AI水文、抄袭博客、低质付费文字、商业广告等无用信息中找到夹缝中的少数高质量文章,这是一件极为痛苦的事情。虽然在少数情况下我们最终能够找到一些有用的信息,但是\textbf{高质量的平台能节约鉴别信息的精力}。

\section{大语言模型和提示词工程}

现在,大语言模型(Large Language Model,LLM)已经广泛地投入了使用,无论是ChatGPT、Claude、Gemini等国外著名LLM,还是国内的DeepSeek、Kimi、通义千问等LLM,都已经投入了广泛应用。LLM使得我们获取信息的方法变得更加简单高效,我们可以把它们当作一个搜索引擎来使用。

\subsection{使用LLM的基本原则}

LLM目前依然只是一个按照概率分布生成文本的模型,而不是一个真正“理解”语言的东西;它的输出是基于统计数据和模式,而不是基于对世界的真正理解。而这个概率除了受到语法、语义等语言学因素的影响和模型本身的影响以外,还受到输入的Prompt(提示词)的影响,因此我们可以通过优化Prompt来在不改进模型性能的条件下尽量优化LLM的输出。这个领域被称为“提示词工程”(Prompt Engineering)。

具体来说,我们要遵循以下原则:

\begin{itemize}
    \item 具体性:使用LLM的时候提问应该极为具体,避免使用模糊、省略的语言或者关键字。例如我们如果想要获取改善睡眠质量的信息,应该使用“如何改善睡眠质量”而不是“改善睡眠”等关键字组合。
    \item 明确性:在使用LLM的时候,我们的Prompt应该明确无歧义。这在LLM上面有一个专门的课题叫做WSD(消歧)。例如“Have a friend for dinner”,我们应该明确地解释成“treat your friend to dinner”或者“Eat your friend”,而不是让LLM去猜。与之类似的是我们可以在Prompt中规定其输出格式,例如提供一个示例,这对获得期望的输出非常有效。
    \item 简单化:目前的AI依然缺乏处理复杂问题的能力。当我们提出一个复杂的问题时,LLM往往会混乱,进而得出错误答案。这时,我们可以采用分治思想,把一个大问题分成多个小问题,然后让LLM分别解决这些小问题,然后合并答案。
\end{itemize}

\subsection{LLM的局限性}

LLM虽然强大,但是它仍然有一些局限性,主要集中在\textbf{信息滞后}与\textbf{幻觉}两大方面。

目前,LLM依然使用的是Transformer架构,这使得它的知识库是静态的。LLM在训练完成之后,其知识库就不会再更新了。这就导致了LLM无法获取最新的信息。虽然有些LLM会定期更新其知识库,但是更新的频率往往较低,且更新的内容也有限。因此,LLM对于截止日期之后的信息往往无法提供准确的答案。而幻觉的来源也很简单:毕竟现在的LLM依然只是在做猜词游戏而已,犯错误非常正常。

所以说,虽然我们将LLM作为一个获取知识的渠道并把它当作一个“超级搜索引擎”来使用,但是我们依然要对其输出内容进行验证,尤其是在其知识库截止日期后的内容:LLM对于一个它不知道的问题往往不会回答“不知道”,而是胡编一个答案出来,这在某些情况下是不可忍受的。

因此,我们在使用LLM的时候,仍然需要保持批判性思维。如果我们希望获取一些旧而笼统的信息,使用LLM能提高我们的搜索效率,并且答案往往是可信的;如果我们希望获取一些新信息或者较为精确的信息,使用LLM是显著不如搜索的。


\section{勇敢地提出问题}

当上述方法全部失败的时候,我们还有最后一个方法:可以抱大佬大腿,或者说向有经验的前辈提问和讨教。

除了抱身边大佬大腿以外,一个最传统的方式是,你可以在上述提到的平台或者其他技术社群上提问相关内容。你可以得到来自不同人的回答,这样你就有概率能够得到更多的帮助。当然,收集到的信息也相对良莠不齐,信息的价值需要自行甄别。同时,你的贴子和问答也会被其他人看到,一定程度上也可造福后人。例如,你可以在Stack Overflow上提出相关技术问题。

另一个方法是在GitHub上发布相关的Issue,这样项目的维护者就会看到你的问题,并提出相关的解答;有时候也有可能是项目本身的问题。这也能够帮助到以后的用户。

在提问的时候,应该遵照以下的原则:

\begin{itemize}
    \item 礼貌与尊重:没有人有义务解答你的问题,解决问题也许会耗费不少的时间和精力,大多数人解答问题往往只是出于本能的善意。礼貌的表达不仅能促使他人更愿意帮助你,还能建立良好的沟通氛围。当下互联网环境下,其实这一点的重要性远超想象。
    \item 增加有用信息:缺乏相关信息会让帮助你的人有心无力。程序崩溃有许多可能情况,不同的情况往往对应着不同的解决方案。如果能够在问题描述中增添足够的有用信息(例如列出错误代码),就会为解决问题增添巨大的可能性。
    \item 减少无用信息:部分人在提问的时候总会无意识地强调与问题无关的东西。这种内容往往会显著地降低信息密度,招致人的反感与厌恶。一个更常见的例子是在社群中发送大段语音而不是文字。
    \item 明确化你的描述:有时,我们的描述会出现歧义或者不明确的现象,例如“直面天命”这个短语对于没有关注或者没有游玩过《黑神话·悟空》的人而言容易导致迷惑。在这种情况下,使用更为具体的“游玩《黑神话·悟空》”等称谓更加合适。
    \item 列出你失败的尝试:这不仅表现出你为了自己解决自己遇到的问题所付出的努力,也能够显著地减少重复劳动与受到类似STFW等回复。
\end{itemize}

一个较好的提问例子是:

“我的电脑突然蓝屏了,我的蓝屏时候遇到的代码是 XXXXXXXX,是在游玩《黑神话·悟空》的时候突然蓝屏的。我上网搜索了代码相关的错误信息,尝试了网上可能有用的 A 方法和 B 方法,但都没有奏效。能麻烦你帮我看看吗?拜托了,非常感谢!”

\end{document}