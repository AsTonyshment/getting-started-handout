\documentclass[../main.tex]{subfiles}

\begin{document}

\chapter{初步使用计算机}

我们已经讲过了计算机的基本构成和工作原理,现在我们来讲一些计算机的初步使用方法。

当然,我们不会涉及到计算机的使用细节,例如如何使用鼠标、如何使用键盘等。我们假设同学们已经具备了最基本的计算机操作能力。这里将会介绍一些科研学习中使用计算机的小技巧,以及一些常用的软件和工具。

\section{维护你的系统}

计算机的维护是一个非常重要的环节。我们需要定期对计算机进行清理和维护,以保证计算机的正常运行。

\subsection{保持更新系统的习惯}

计算机在运行过程中,操作系统和软件会不断地更新,以修复漏洞、提高性能和增加新功能。我们应该定期检查系统和软件的更新,并及时按需要安装它们。

对于一些重要的更新(例如安全更新等),我们应该立即安装,这是因为此类更新通常是为了修复一些新近发现的漏洞和问题,如果不及时安装,可能会导致计算机被攻击或者出现其他问题。而对于一些不重要的更新(例如功能更新等),我们可以根据自己的需要选择安装。

特别注意:虽然我们提倡保持更新,但是在生产类环境中,贸然更新可能会导致系统不稳定或软件不兼容。因此,在生产环境中,我们应该在更新之前进行充分的测试,确保更新不会影响系统的正常运行,或者使用虚拟机等隔离环境运行生产用代码。

\subsection{定期备份数据}

定期备份数据是保护计算机数据安全的重要措施。我们可以使用外部硬盘、云存储等方式备份数据,以防止信息泄露或者重要文件丢失。数据备份的频率可以根据数据的重要性和变化频率来决定。

数据备份有一个重要的原则:\textbf{3-2-1备份法则}。即:至少保留三份数据备份,存储在两个不同的介质上,其中一份存储在异地。例如,我们可以在本地硬盘上存储一份数据备份,在外部硬盘上存储一份数据备份,并将另一份数据备份存储在云端。这样,即使其中一份甚至两份损坏或者丢失,我们也可以通过其他方式恢复数据。

\subsection{定期清理系统}

定期清理系统可以提高计算机的性能和安全性。我们可以使用一些系统清理工具,删除不必要的文件、缓存和临时文件等,以释放磁盘空间和提高系统性能。我们推荐使用系统自带的清理工具,例如Windows的磁盘清理工具、macOS的存储管理工具等。如果较为富裕,也可以使用一些知名的清理软件,例如CCleaner等(免费版已经足够好用了)。出于众所周知的原因,我们不推荐使用360等软件。

除此之外,休眠文件、系统还原点等也会占用大量磁盘空间。我们可以根据自己的需要,选择是否保留这些文件。

特别注意的是,清理系统和减肥差不多,同样需要\textbf{缓慢、谨慎、循序渐进}地进行。

\subsection{碎片整理}

在计算机使用过程中,如果使用机械硬盘,文件的删除和修改会导致磁盘上的数据变得零散,从而影响计算机的性能。我们可以使用碎片整理工具,定期对磁盘进行碎片整理(即重排文件使其连续),以部分提高磁盘的读写速度。

直接使用Windows自带的碎片整理工具即可。对于固态硬盘,碎片整理并不会提高性能,反而会缩短使用寿命,因此不建议对SSD进行碎片整理。

\section{善用快捷键}

使用快捷键可以减少鼠标操作的频率。以下是来自Windows的常用快捷键:
\begin{itemize}
    \item Ctrl+C:复制
    \item Ctrl+V:粘贴
    \item Ctrl+Z:撤销
    \item Ctrl+Y:重做
    \item Ctrl+A:全选
    \item Ctrl+Alt+Del:任务管理器
    \item Alt+F4:关闭窗口
    \item Win+R:打开运行窗口
    \item Win+E:文件资源管理器\footnote{文件资源管理器是Windows系统中用于浏览和管理文件和文件夹的工具。}
    \item Win+D:显示桌面
    \item Win+L:锁定计算机
    \item Ctrl+S:保存
    \item Ctrl+P:打印
    \item Ctrl+F:查找替换
\end{itemize}

\section{终端初步}\label{sec:terminal}

终端是计算机与操作系统之间的一个交互界面。它允许用户通过命令行输入指令,与操作系统进行交互。

虽然终端是一个非常古老的工具,但是使用它依然可以提高工作效率,尤其是在处理大量文件或者进行复杂操作时。它还可以用于远程连接到其他计算机,进行远程管理和维护等操作。

对于Linux和macOS,我们建议使用Zsh或Fish等更加友好的shell,他们能够进行语法高亮、自动补全等操作。对于Windows,我们建议使用Windows PowerShell。PowerShell的命令统一采用的是动词-名词的格式,和Linux Shell的简单缩写形式有很大的不同。这是因为PowerShell的设计理念是模仿C\#的“对象”,而不是Linux Shell的文本流。不过也正因此,PowerShell本身就是一门完备的语言,功能非常强大,在处理复杂的任务上更为简单。

\begin{table}[htbp]
    \centering
    \begin{tabular}{|c|cc|}
        \hline
        \textbf{操作} & \textbf{Bash} & \textbf{Pwsh} \\
        \hline
        创建文件 & touch & New-Item \\
        \hline
        列出文件 & ls & Get-ChildItem \\
        \hline
        复制文件 & cp & Copy-Item \\
        \hline
        移动文件 & mv & Move-Item \\
        \hline
        删除文件 & rm & Remove-Item \\
        \hline
        创建目录 & mkdir & New-Item -Type Directory \\
        \hline
        删除目录 & rmdir & Remove-Item -Recurse \\
        \hline
        查看帮助 & man & Get-Help \\
        \hline
    \end{tabular}
    \caption{Bash和PowerShell的常用命令对比}
    \label{tab:terminal-commands}
\end{table}

这仅仅是最基本的命令使用方法,实际上终端的命令系统非常复杂,功能也非常强大。感兴趣的同学可以自行查找相关资料进一步学习。


\section{Git初步}\label{sec:git}

试想以下环境:我们正在写一项作业,开发工作已经基本完成,试运行也能够得到90分。此时我们希望进一步精进代码,使得分数达到95分以上;但是经过一通修改以后,发现程序再也运行不起来了。这时候距离ddl只有1小时,我们决定摆烂,提交能够得到90分的代码。然后我们根据记忆改回原来的代码的时候,发现我们再也想不起来旧代码是怎么写的了!这无疑是令人极为懊恼的。

为了避免以上问题,我们引入了版本控制系统。目前最常用的版本控制系统是Git。

\subsection{Git的工作原理}

Git有三个目录共同完成版本控制:工作区、暂存区、版本库。工作区是项目目录,暂存区是一个隐藏的文件夹.git,版本库是一个隐藏的文件夹.git/objects。工作区是我们平时使用的目录,暂存区是Git用来存储修改的地方,版本库是Git用来存储所有版本信息的地方。版本库有一个指针,指向当前版本的某一节点(一般指向最新的节点)。每个节点都有一个唯一的哈希值,用来标识该节点。每个节点包含了该版本的所有文件和目录的信息,以及指向上一个版本的指针。Git使用哈希值来标识每个版本,这样可以保证每个版本都是唯一的。

这样讲解很难以理解,我们不妨举例说明:现在,Git中有一个版本为X的节点,包括文件A和文件B两个文件。这些文件存储在版本库中。此时,工作区为空,暂存区为空,指针指向X。我现在希望对它们进行修改,这个修改遵循以下过程:

\begin{enumerate}
    \item 我拿出了这些文件,并且对文件A进行修改。此时,工作区有AB两个文件,但是暂存区依然是空的。我们的任何修改都不会被暂存区记录,Git也不会知道我对这些文件进行了修改。
    \item 我觉得修改差不多了,现在把A放进暂存区。现在Git知道我对A进行了一些修改了。
    \item 我又对B进行了类似的修改,此时B也进暂存区了。
    \item 我觉得修改差不多了。我认为我应该永久保存目前的状态,于是就把暂存区提交到版本库。此时版本库多了一个Y节点,指针也指向Y节点,有修改过的AB两个文件。此时,暂存区又清空了,而工作区和版本库的Y版本一致。
\end{enumerate}

\subsection{下载Git}

一个最简单的方式是使用Winget包管理器:

\begin{verbatim}
    winget install Microsoft.git
\end{verbatim}

或者你也可以从官方网站上下载并安装之。同样,安装的时候一定要勾选“添加到PATH”这一选项,否则你在命令行中无法使用Git。

\subsection{Git信息设置}

安装并使用Git的第一步是先编辑本地的一些提交信息。Git的提交需要一个用户名和一个邮箱,来对应每次提交的作者。我们可以使用以下命令来设置这些信息:

\begin{verbatim}
    git config --global user.name "Your Name"
    git config --global user.email "email@example.com"
\end{verbatim}

这样即可设置全局用户名和邮箱。如希望给某个特定仓库设置特定的用户名和邮箱,你需要在该仓库下重新执行上述命令,但是不写--global命令。

现代Git一般提倡使用main作为根分支的名称。而Git依然使用旧的master分支作为根分支,你可以使用以下命令修改为main:

\begin{verbatim}
    git config --global init.defaultBranch main 
    # 这条命令会修改全局的默认分支名称
\end{verbatim}

\subsection{Git的最基本使用}

\subsubsection{版本控制:提交}

要具体地在某一目录下进行版本控制,我们需要在命令行中进入到我们希望使用Git的目录下。然后我们可以使用以下命令来初始化一个Git仓库:

\begin{verbatim}
    git init
\end{verbatim}

如果你在视窗中开启了“显示隐藏文件”这类功能,你就会发现一个隐藏的文件夹.git出现在了你当前的目录下。这个文件夹就是Git用来存储版本信息的地方。

然后你可以使用以下命令来添加文件到Git仓库中(这个命令的实际意义是把文件添加到暂存区);

\begin{verbatim}
    git add <filename>
\end{verbatim}

如果我们忘记了当前状态下有哪些文件被修改了,我们可以使用以下命令来查看当前状态:
\begin{verbatim}
    git status
\end{verbatim}

如果你觉得修改差不多了,保存文件以后,你可以使用以下命令来提交文件到Git仓库中(这个命令的实际意义是把暂存区的文件提交到版本库中):

\begin{verbatim}
    git commit -m "commit message"
\end{verbatim}

上述内容中,-m后面是提交信息。提交信息是对本次提交的简要描述。我们建议每次提交都写上简要的提交信息,这样可以帮助我们更好地理解代码的修改历史。

\subsubsection{版本控制:回退}

如果出现了先前我们说的不小心写坏了的情况,这时候就可以进行版本回退了。我们可以使用以下命令来查看当前的版本信息:

\begin{verbatim}
    git log # 例如版本库是a-b-c-d-e-f-g
\end{verbatim}

找到你希望回退到的版本的哈希值(前几位即可),然后使用以下命令来回退到该版本(这个命令会把指针回退到指定的版本,丢弃之后的所有内容,然后丢弃暂存区和工作区的所有东西):

\begin{verbatim}
    git reset --hard <commit_hash> 
    # 请谨慎使用这一命令!该命令不会保留当前的修改!
\end{verbatim}

如果你希望回退到某个版本,但是不想丢失当前的修改,你可以使用以下命令来回退到该版本(这个命令会把版本库后面的东西全部丢弃,清空暂存区,但是保留当前工作区):
\begin{verbatim}
    git reset --mixed <commit_hash> 
    # 我们更加推荐这个回退方式,--mixed可以省略,或者用--soft替代。
    用--soft替代时,不会清空暂存区。
\end{verbatim}

\subsubsection{版本控制:排除相关文件}

有时候我们版本跟踪的时候不需要跟踪一些文件,例如具有敏感信息的文件(如密码),或者构建文件等。此时,我们可以创建一个文件 .gitignore 来阻止跟踪。例如,在Linux下,构建文件往往是*.o。那么我们可以在上述文件中加入 *.o ,之后git就会忽略这些文件。

关于Git版本控制的一些更加进阶的知识(例如分支管理等内容),欢迎查阅更多资料。我们在高阶课程\ref{sec:git-advanced}中会介绍一些Git的进阶用法。

\end{document}